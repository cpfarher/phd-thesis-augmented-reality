\chapter*{Resumen}
\addcontentsline{toc}{chapter}{Resumen}
La detección de objetos es un área de estudio creciente y ha dado lugar a aplicaciones relevantes en diversas áreas (robótica \cite{conf/icra/2010}, vigilancia, \cite{5672610}, realidad aumentada \cite{5739718}, etc.), sin embargo aún queda mucho por explorar. En este trabajo, se presenta un método para la detección de objetos planos en cualquier perspectiva sin emplear marcadores, con procesamiento en tiempo real sobre flujo de video. Luego, se aplica en un prototipo de realidad aumentada.

El método desarrollado utiliza un detector y descriptor de características del estado del arte denominado SURF \cite{Bay:2008:SRF}. Para encontrar la posición del objeto se usa la homografía que se genera mediante la búsqueda de correspondencias de características entre una imagen patrón y el flujo de video, obtenido en tiempo real utilizando una cámara web de una computadora portátil. A los pasos generales propuestos en diferentes trabajos, aquí se le agregan etapas para mejorar la detección aumentando el rendimiento en cuanto a la velocidad y una serie de validaciones para evitar detecciones incorrectas y homografías que produzcan transformaciones defectuosas. También se proponen algunas técnicas sencillas para incrementar los detalles y mejorar la iluminación, a partir de experimentos y análisis de su comportamiento.

Finalmente se presentan dos prototipos: el primero localiza una imagen patrón en la escena, la cual puede presentarse escalada, rotada, y/o en perspectiva, y luego se sobrepone una fotografía en la región donde se detectó el patrón. El segundo prototipo fue desarrollado para una aplicación publicitaria, utilizando como imagen de patrón el packaging de un producto alimenticio y aplicando realidad aumentada para brindar información adicional del mismo.