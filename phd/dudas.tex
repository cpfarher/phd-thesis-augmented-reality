\chapter{Preguntas y aclaraciones}
\label{c:parte6}

\vspace{1cm}

\newpage

\section{Transormación AFIN}
Para áreas planas hay dos sabores de transformaciones geométricas: las que usan matrices de 2x3 que son llamadas \textbf{transformaciones afines}; y las transformaciones basadas en matrices de 3x3 que se llaman \textbf{transformaciones perspectivas u homografias}.

Una \textbf{transformación afín} es cualquier transformación que pude ser expresada como una multiplicación de matrices seguida de la adición de un vector. Las transformaciones afines pueden convertir rectángulos en paralelogramos (los lados se mantienen paralelos, la figura puede rotar o escalarse).

Las transformaciones perspectivas ofrecen más flexibilidad pudiendo transformar un rectángulo en un trapezoide y como un paralelogramo es también un trapezoide, una transformación afín es un subconjunto de las transformaciones perspectivas.

\section{Porqué para mejoramiento de la iluminación no se uso filtro homomórfico?}
El filtro homomórfico (FH) simultáneamente normaliza el brillo e incrementa el contraste. Pero utiliza la transformada de Fourier para pasar al dominio de las frecuencias y luego la inversa de la transformada. Esto lo hace lento cuando se lo quiere aplicar y por ello no se tuvo en cuenta en el trabajo.

\section{Error de reproyección y cantidad de iteraciones}
en la fórmula se usan 5 píxeles.

\subsection{En opencv el que usé}
a
\subsection{En la teoría}
b
\subsection{En los libros}
c
\section{Cuan bien realiza la ubicación el método}
a
\section{Matriz hessiana}
Si se tiene una matriz de n variables, se obtiene una matriz hessiana de $n \times n$.

\textbf{Traza de la matriz:} En álgebra lineal, la traza de una matriz cuadrada A de nxn está definida como la suma de los elementos de la diagonal principal de A. $tr(A)=a_{11}+a_{22}+a_{33}+...+a_{nn}$.

Ejemplo de calculo del determinante hessiano: \url{http://blog.utp.edu.co/alejandropinto/files/2010/11/Discriminante-o-Hessiano.pdf}.

\textbf{Determinante de la hessiana como determinar el umbral}

\subsection{surf porque y donde detecta los puntos}
Result of local nonmax suppression of $det(H_{approx})$ en una vecindad de $3\times3\times3$ del punto.
\subsection{surf deteccion puntos en el espacio escala}
a
\section{cuantos grados de libertad tiene el proyecto}

\section{Que tipo de deformaciones fotométricas se asumen}

\section{Donde se calculan las wavelets?}

\section{parámetros extrínsecos e intrínsecos míos? o en H?}
\begin{itemize}
  \item \textbf{Extrínsecos:} $[R, t]$ posición y orientación respecto a la escena.
  \item \textbf{Intrínsecos:} 
    \begin{itemize}
      \item características propias físicas y ópticas de la cámara.
      \item distancia focal
      \item la dimensión imagen y escala
      \item el punto principal en la imagen en unidades de modelo 
      \item se ignora la distorsión provocada por la lente
      \item la misma cámara en diferentes posiciones tienen los mismo parámetros intrínsecos.
    \end{itemize}
\end{itemize}

\section{imagenes integrales}
\url{http://www.aishack.in/2010/07/integral-images/}

\section{Homografía iteraciones}
\url{http://www.cc.gatech.edu/~richard/ransacld4/}.
0.8 es la distancia establecida para el PnP, con 0.6 mas rápido pero menos estable.
Esta a 5 píxeles de distancia el reprojection error de ransactReprojThreshold: cvfindhomography

\textbf{ransacReprojThreshold (Double)}
The maximum allowed reprojection error to treat a point pair as an inlier. The parameter is only used in RANSAC-based homography estimation. E.g. if dst\_points coordinates are measured in pixels with pixel-accurate precision, it makes sense to set this parameter somewhere in the range $~1..3$.

Imágenes integrales -> si bien el tamaño no lo afecta, el cálculo de la imágen integral si lo afecta... => por tanto la detección de movimiento en un pedazo de la imágen se ve beneficidad porque el cálculo de la imágen integral es menor...
\section{Flujo óptico}
La representación bidimensional de movimiento tridimensional se denomina campo de movimiento, en donde a cada punto se le asigna un vector de velocidad correspondiente a la dirección de movimiento, velocidad y ubicación en un lugar específico de la imagen.

El flujo óptico refleja los cambios en la imagen debidos al movimiento ocurrido en un diferencial de tiempo $\delta t$, que en el caso de secuencias discretas usualmente corresponde a dos imágenes adyacentes en la secuencia.

En principio, se aspira a tener detectores de flujo óptico insensible a cambios de iluminación o a objetos irrelevantes (como sombras). Sin embargo, las técnicas usuales detectan movimiento, por ejemplo, si una fuente de luz se mueve sobre una esfera con superficie especular, y no detectan movimiento si la fuente de luz es constante y la esfera rota sobre su centro.

El cálculo del flujo óptico parte de dos suposiciones:
\begin{itemize}
 \item El brillo percibido de cualquier punto de un objeto permanece aproximadamente
constante en el tiempo.
 \item Puntos cercanos en el plano de la imagen se mueven de forma similar (restricción de suavidad en la velocidad).
\end{itemize}

\section{linea 830}
Linea 830: Luego del punto decía: ``Las repuestas wavelet son invariantes a un sesgo en la iluminación, mientras que la invarianza al contraste es alcanzado mediante la normalización del descriptor''. Eso lo decía en el paper original y lo saqué de la tesis porque no se bien a que se refiere... esta bien? o lo dejo igual?

ok, nadie te va a preguntar. Creo q indica que las wavelet son robustas a cambios pequeños en iluminación, y lo otro normaliza el descriptor entonces el contraste le quedará en la misma escala p la comparación y no tiene mucha varianza entre uno y otro calculo.
