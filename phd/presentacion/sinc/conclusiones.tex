\section{Conclusiones}
\begin{frame}{Conclusiones}
 \begin{itemize}
 \item Afianzamiento de los conocimientos en el área de visión computacional.
 \item Se presentó un método para detectar objetos planos (sin marcadores) en un ambiente controlado, mediante la detección de características sobre imágenes obtenidas con una cámara web de computadora,
 \item Introducción de conceptos de RA, detectores de características relevantes y herramientas para detección y seguimiento de objetos,
 \end{itemize}
\end{frame}
%
\begin{frame}{Conclusiones}
 \begin{itemize}
 \item Técnicas simples para realce de detalles e iluminación, de forma que el usuario pueda considerar la aplicación de alguna de ellas de ser necesario,
 \item Aplicación de estrategias y optimización para una correcta detección, sin afectar al tiempo de procesamiento de forma considerable,
 \item Codificación en C++ y utilización de librería OpenCV que permiten la portabilidad a diferentes sistemas operativos,
 \end{itemize}
\end{frame}
%%%%%%%%%%%%%%%%%%%%%%%%%%%%%%%%%%%%%%%%%%%%%%%%%%%%%%%%%%%%%%%%%%%%%%%%%%%%%%%%%%%%%%%%%%%%%%%%%%%%%%%%%%%%%%%%%%%%%%%%%%%%%%%%%%%%%%%%%
\begin{frame}{Trabajos futuros}
\textbf{Corto Plazo:}
\begin{itemize}
 \item GPU y múltiples hilos de procesamiento aplicado al proceso de extracción y descripción de características.
 \item Utilizar para la búsqueda de correspondencias ``nano flann'' en vez de FLANN: eficiencia en tiempo y consumo de memoria.
 \item Selección automática del umbral hessiano para la discriminación de puntos.
\end{itemize}
\end{frame}
%
\begin{frame}{Trabajos futuros}
\textbf{Mediano Plazo:}
\begin{itemize}
 \item Usar otros detectores en el proceso de extracción y descripción de características: MSER, FAST, etc. Comparación en \url{http://bit.ly/hox3gW}.
 \item Investigar e implementar soluciones para el manejo de la oclusión.
\end{itemize}
\end{frame}
%
\begin{frame}{Trabajos futuros}
\textbf{Largo Plazo:}
\begin{itemize}
 \item Implementar una aplicación con interfaz gráfica orientada a un usuario final. Reconocer diferentes imágenes patrón de una BD y llevar a cabo una acción particular para el enriquecimiento de la realidad.
 \item Implementar el sistema en dispositivos móviles (tablets, celulares, etc.).% utilizando ``procesamiento en la nube (cloud computing)''.
\end{itemize}
\end{frame}